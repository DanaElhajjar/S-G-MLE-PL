\begin{abstract}
This paper introduces a novel approach for sequential estimation of the interferometric phase in the context of  long \ac{SAR} image time series. When newly acquired data arrive, the data set expands and can be partitioned into two distinct blocks. One represents the previous \acs{SAR} images and the other represents the newly acquired data. The proposed approach (S-MLE-PL) exploits sequential maximum likelihood estimation of the covariance matrix of the whole data set, % in a Bayesian framework,
taking the existing data set as prior information. This approach facilitates the continuous interferometric phase estimation by incorporating the new data into  the previous context. In addition, it presents the advantage of reduced computation time compared to the traditional approaches, making it a more efficient solution for operational displacement estimation.
\end{abstract}

\begin{keywords}
Multi-temporal InSAR, sequential estimation, covariance matrix, phase linking
\end{keywords}

