\vspace{-5pt} 
\section{Sequential Phase Linking}
\label{sec:approach}
\vspace{-5pt} 
We consider a stack of $l = p+1$ \acs{SAR} images, for a given pixel, we denote $\{\mathbf{\tilde{x}}^i\}_{i=1}^n$ the local homogeneous spatial neighborhood of size $n$, where $\mathbf{\tilde{x}}^i \in \mathbb{C}^{l}$, for all $i \in [\![1,n]\!]$, i.e., 
\vspace{-7pt} 
\begin{equation}
    \mathbf{\tilde{x}}^i = [\underbrace{x_1^i, \dots, x_p^i}_{\mathbf{x}^i}, x_{l}^i ]^T \in \mathbb{C}^{l}
\end{equation}
where $\mathbf{x}^i \in \mathbb{C}^{p}$ denotes the multivariate pixel of the previous data (Fig. \ref{fig:datacube}). Each pixel of the local patch is assumed to be distributed as a zero mean \ac{CCG} \citep{bamler1998synthetic}, i.e., $\mathbf{\tilde{x}}\sim\mathcal{CN}(0, \mathbf{\tilde{\Sigma}})$. 

\begin{figure}[hbt]
    \centering
\resizebox{\linewidth}{!}{\begin{tikzpicture}[x=0.4pt,y=0.4pt,yscale=-1,xscale=1]
%uncomment if require: \path (0,927); %set diagram left start at 0, and has height of 927
%Shape: Cube [id:dp30838284267869476] 
\draw  [fill={rgb, 255:red, 255; green, 255; blue, 255 }  ,fill opacity=1 ] (456.91,238.11) -- (480.57,214.46) -- (495.19,214.46) -- (495.19,229.9) -- (471.53,253.56) -- (456.91,253.56) -- cycle ; \draw   (495.19,214.46) -- (471.53,238.11) -- (456.91,238.11) ; \draw   (471.53,238.11) -- (471.53,253.56) ;
%Shape: Cube [id:dp3902884736266059] 
\draw  [fill={rgb, 255:red, 255; green, 255; blue, 255 }  ,fill opacity=1 ] (456.91,219.87) -- (480.57,196.21) -- (495.19,196.21) -- (495.19,211.65) -- (471.53,235.31) -- (456.91,235.31) -- cycle ; \draw   (495.19,196.21) -- (471.53,219.87) -- (456.91,219.87) ; \draw   (471.53,219.87) -- (471.53,235.31) ;
%Shape: Cube [id:dp3890748389560592] 
\draw  [fill={rgb, 255:red, 255; green, 255; blue, 255 }  ,fill opacity=1 ] (456.91,201.62) -- (480.57,177.96) -- (495.19,177.96) -- (495.19,193.41) -- (471.53,217.06) -- (456.91,217.06) -- cycle ; \draw   (495.19,177.96) -- (471.53,201.62) -- (456.91,201.62) ; \draw   (471.53,201.62) -- (471.53,217.06) ;
%Shape: Cube [id:dp8092639219693263] 
\draw  [fill={rgb, 255:red, 255; green, 255; blue, 255 }  ,fill opacity=1 ] (456.91,183.37) -- (480.57,159.72) -- (495.19,159.72) -- (495.19,175.16) -- (471.53,198.82) -- (456.91,198.82) -- cycle ; \draw   (495.19,159.72) -- (471.53,183.37) -- (456.91,183.37) ; \draw   (471.53,183.37) -- (471.53,198.82) ;
%Shape: Cube [id:dp8821698787582473] 
\draw  [fill={rgb, 255:red, 255; green, 255; blue, 255 }  ,fill opacity=1 ] (456.91,165.13) -- (480.57,141.47) -- (495.19,141.47) -- (495.19,156.91) -- (471.53,180.57) -- (456.91,180.57) -- cycle ; \draw   (495.19,141.47) -- (471.53,165.13) -- (456.91,165.13) ; \draw   (471.53,165.13) -- (471.53,180.57) ;
%Shape: Cube [id:dp05048815941659335] 
\draw  [fill={rgb, 255:red, 255; green, 255; blue, 255 }  ,fill opacity=1 ] (456.91,146.88) -- (480.57,123.22) -- (495.19,123.22) -- (495.19,138.67) -- (471.53,162.32) -- (456.91,162.32) -- cycle ; \draw   (495.19,123.22) -- (471.53,146.88) -- (456.91,146.88) ; \draw   (471.53,146.88) -- (471.53,162.32) ;
%Shape: Cube [id:dp41289533406576906] 
\draw  [fill={rgb, 255:red, 255; green, 255; blue, 255 }  ,fill opacity=1 ] (338.25,265.48) -- (361.91,241.83) -- (376.53,241.83) -- (376.53,257.27) -- (352.87,280.93) -- (338.25,280.93) -- cycle ; \draw   (376.53,241.83) -- (352.87,265.48) -- (338.25,265.48) ; \draw   (352.87,265.48) -- (352.87,280.93) ;
%Shape: Cube [id:dp2305106364033589] 
\draw  [fill={rgb, 255:red, 255; green, 255; blue, 255 }  ,fill opacity=1 ] (356.7,265.48) -- (380.36,241.83) -- (394.98,241.83) -- (394.98,257.27) -- (371.32,280.93) -- (356.7,280.93) -- cycle ; \draw   (394.98,241.83) -- (371.32,265.48) -- (356.7,265.48) ; \draw   (371.32,265.48) -- (371.32,280.93) ;
%Shape: Cube [id:dp7583591419148019] 
\draw  [fill={rgb, 255:red, 255; green, 255; blue, 255 }  ,fill opacity=1 ] (375.15,265.48) -- (398.8,241.83) -- (413.42,241.83) -- (413.42,257.27) -- (389.77,280.93) -- (375.15,280.93) -- cycle ; \draw   (413.42,241.83) -- (389.77,265.48) -- (375.15,265.48) ; \draw   (389.77,265.48) -- (389.77,280.93) ;
%Shape: Cube [id:dp9049896144069223] 
\draw  [fill={rgb, 255:red, 255; green, 255; blue, 255 }  ,fill opacity=1 ] (393.12,265.48) -- (416.77,241.83) -- (431.39,241.83) -- (431.39,257.27) -- (407.73,280.93) -- (393.12,280.93) -- cycle ; \draw   (431.39,241.83) -- (407.73,265.48) -- (393.12,265.48) ; \draw   (407.73,265.48) -- (407.73,280.93) ;
%Shape: Cube [id:dp09448256983641268] 
\draw  [fill={rgb, 255:red, 255; green, 255; blue, 255 }  ,fill opacity=1 ] (338.25,247.24) -- (361.91,223.58) -- (376.53,223.58) -- (376.53,239.02) -- (352.87,262.68) -- (338.25,262.68) -- cycle ; \draw   (376.53,223.58) -- (352.87,247.24) -- (338.25,247.24) ; \draw   (352.87,247.24) -- (352.87,262.68) ;
%Shape: Cube [id:dp5602366258220954] 
\draw  [fill={rgb, 255:red, 255; green, 255; blue, 255 }  ,fill opacity=1 ] (356.7,247.24) -- (380.36,223.58) -- (394.98,223.58) -- (394.98,239.02) -- (371.32,262.68) -- (356.7,262.68) -- cycle ; \draw   (394.98,223.58) -- (371.32,247.24) -- (356.7,247.24) ; \draw   (371.32,247.24) -- (371.32,262.68) ;
%Shape: Cube [id:dp3712353447092802] 
\draw  [fill={rgb, 255:red, 255; green, 255; blue, 255 }  ,fill opacity=1 ] (375.15,247.24) -- (398.8,223.58) -- (413.42,223.58) -- (413.42,239.02) -- (389.77,262.68) -- (375.15,262.68) -- cycle ; \draw   (413.42,223.58) -- (389.77,247.24) -- (375.15,247.24) ; \draw   (389.77,247.24) -- (389.77,262.68) ;
%Shape: Cube [id:dp055132731096200205] 
\draw  [fill={rgb, 255:red, 255; green, 255; blue, 255 }  ,fill opacity=1 ] (393.12,247.24) -- (416.77,223.58) -- (431.39,223.58) -- (431.39,239.02) -- (407.73,262.68) -- (393.12,262.68) -- cycle ; \draw   (431.39,223.58) -- (407.73,247.24) -- (393.12,247.24) ; \draw   (407.73,247.24) -- (407.73,262.68) ;
%Shape: Cube [id:dp313251615103586] 
\draw  [fill={rgb, 255:red, 255; green, 255; blue, 255 }  ,fill opacity=1 ] (338.25,228.99) -- (361.91,205.33) -- (376.53,205.33) -- (376.53,220.78) -- (352.87,244.43) -- (338.25,244.43) -- cycle ; \draw   (376.53,205.33) -- (352.87,228.99) -- (338.25,228.99) ; \draw   (352.87,228.99) -- (352.87,244.43) ;
%Shape: Cube [id:dp5827550641318961] 
\draw  [fill={rgb, 255:red, 255; green, 255; blue, 255 }  ,fill opacity=1 ] (356.7,228.99) -- (380.36,205.33) -- (394.98,205.33) -- (394.98,220.78) -- (371.32,244.43) -- (356.7,244.43) -- cycle ; \draw   (394.98,205.33) -- (371.32,228.99) -- (356.7,228.99) ; \draw   (371.32,228.99) -- (371.32,244.43) ;
%Shape: Cube [id:dp6327172987649727] 
\draw  [fill={rgb, 255:red, 255; green, 255; blue, 255 }  ,fill opacity=1 ] (375.15,228.99) -- (398.8,205.33) -- (413.42,205.33) -- (413.42,220.78) -- (389.77,244.43) -- (375.15,244.43) -- cycle ; \draw   (413.42,205.33) -- (389.77,228.99) -- (375.15,228.99) ; \draw   (389.77,228.99) -- (389.77,244.43) ;
%Shape: Cube [id:dp3948275231630143] 
\draw  [fill={rgb, 255:red, 255; green, 255; blue, 255 }  ,fill opacity=1 ] (393.12,228.99) -- (416.77,205.33) -- (431.39,205.33) -- (431.39,220.78) -- (407.73,244.43) -- (393.12,244.43) -- cycle ; \draw   (431.39,205.33) -- (407.73,228.99) -- (393.12,228.99) ; \draw   (407.73,228.99) -- (407.73,244.43) ;
%Shape: Cube [id:dp5197137900999884] 
\draw  [fill={rgb, 255:red, 255; green, 255; blue, 255 }  ,fill opacity=1 ] (338.25,210.74) -- (361.91,187.09) -- (376.53,187.09) -- (376.53,202.53) -- (352.87,226.19) -- (338.25,226.19) -- cycle ; \draw   (376.53,187.09) -- (352.87,210.74) -- (338.25,210.74) ; \draw   (352.87,210.74) -- (352.87,226.19) ;
%Shape: Cube [id:dp7358007860704894] 
\draw  [fill={rgb, 255:red, 255; green, 255; blue, 255 }  ,fill opacity=1 ] (356.7,210.74) -- (380.36,187.09) -- (394.98,187.09) -- (394.98,202.53) -- (371.32,226.19) -- (356.7,226.19) -- cycle ; \draw   (394.98,187.09) -- (371.32,210.74) -- (356.7,210.74) ; \draw   (371.32,210.74) -- (371.32,226.19) ;
%Shape: Cube [id:dp5902020550600033] 
\draw  [fill={rgb, 255:red, 255; green, 255; blue, 255 }  ,fill opacity=1 ] (375.15,210.74) -- (398.8,187.09) -- (413.42,187.09) -- (413.42,202.53) -- (389.77,226.19) -- (375.15,226.19) -- cycle ; \draw   (413.42,187.09) -- (389.77,210.74) -- (375.15,210.74) ; \draw   (389.77,210.74) -- (389.77,226.19) ;
%Shape: Cube [id:dp6233304389034731] 
\draw  [fill={rgb, 255:red, 255; green, 255; blue, 255 }  ,fill opacity=1 ] (393.12,210.74) -- (416.77,187.09) -- (431.39,187.09) -- (431.39,202.53) -- (407.73,226.19) -- (393.12,226.19) -- cycle ; \draw   (431.39,187.09) -- (407.73,210.74) -- (393.12,210.74) ; \draw   (407.73,210.74) -- (407.73,226.19) ;
%Shape: Cube [id:dp019859677909012996] 
\draw  [fill={rgb, 255:red, 255; green, 255; blue, 255 }  ,fill opacity=1 ] (410.98,265.48) -- (434.63,241.83) -- (449.25,241.83) -- (449.25,257.27) -- (425.6,280.93) -- (410.98,280.93) -- cycle ; \draw   (449.25,241.83) -- (425.6,265.48) -- (410.98,265.48) ; \draw   (425.6,265.48) -- (425.6,280.93) ;
%Shape: Cube [id:dp30966220319052407] 
\draw  [fill={rgb, 255:red, 255; green, 255; blue, 255 }  ,fill opacity=1 ] (429.42,265.48) -- (453.08,241.83) -- (467.7,241.83) -- (467.7,257.27) -- (444.04,280.93) -- (429.42,280.93) -- cycle ; \draw   (467.7,241.83) -- (444.04,265.48) -- (429.42,265.48) ; \draw   (444.04,265.48) -- (444.04,280.93) ;
%Shape: Cube [id:dp9133903549524531] 
\draw  [fill={rgb, 255:red, 255; green, 255; blue, 255 }  ,fill opacity=1 ] (447.87,265.48) -- (471.53,241.83) -- (486.15,241.83) -- (486.15,257.27) -- (462.49,280.93) -- (447.87,280.93) -- cycle ; \draw   (486.15,241.83) -- (462.49,265.48) -- (447.87,265.48) ; \draw   (462.49,265.48) -- (462.49,280.93) ;
%Shape: Cube [id:dp9506675404469944] 
\draw  [fill={rgb, 255:red, 255; green, 255; blue, 255 }  ,fill opacity=1 ] (465.84,265.48) -- (489.5,241.83) -- (504.12,241.83) -- (504.12,257.27) -- (480.46,280.93) -- (465.84,280.93) -- cycle ; \draw   (504.12,241.83) -- (480.46,265.48) -- (465.84,265.48) ; \draw   (480.46,265.48) -- (480.46,280.93) ;
%Shape: Cube [id:dp8696397264388416] 
\draw  [fill={rgb, 255:red, 255; green, 255; blue, 255 }  ,fill opacity=1 ] (410.98,247.24) -- (434.63,223.58) -- (449.25,223.58) -- (449.25,239.02) -- (425.6,262.68) -- (410.98,262.68) -- cycle ; \draw   (449.25,223.58) -- (425.6,247.24) -- (410.98,247.24) ; \draw   (425.6,247.24) -- (425.6,262.68) ;
%Shape: Cube [id:dp9918644205108811] 
\draw  [fill={rgb, 255:red, 255; green, 255; blue, 255 }  ,fill opacity=1 ] (429.42,247.24) -- (453.08,223.58) -- (467.7,223.58) -- (467.7,239.02) -- (444.04,262.68) -- (429.42,262.68) -- cycle ; \draw   (467.7,223.58) -- (444.04,247.24) -- (429.42,247.24) ; \draw   (444.04,247.24) -- (444.04,262.68) ;
%Shape: Cube [id:dp17502558953866054] 
\draw  [fill={rgb, 255:red, 255; green, 255; blue, 255 }  ,fill opacity=1 ] (447.87,247.24) -- (471.53,223.58) -- (486.15,223.58) -- (486.15,239.02) -- (462.49,262.68) -- (447.87,262.68) -- cycle ; \draw   (486.15,223.58) -- (462.49,247.24) -- (447.87,247.24) ; \draw   (462.49,247.24) -- (462.49,262.68) ;
%Shape: Cube [id:dp8691487385154433] 
\draw  [fill={rgb, 255:red, 255; green, 255; blue, 255 }  ,fill opacity=1 ] (465.84,247.18) -- (489.5,223.52) -- (504.12,223.52) -- (504.12,238.97) -- (480.46,262.62) -- (465.84,262.62) -- cycle ; \draw   (504.12,223.52) -- (480.46,247.18) -- (465.84,247.18) ; \draw   (480.46,247.18) -- (480.46,262.62) ;
%Shape: Cube [id:dp6456925082067195] 
\draw  [fill={rgb, 255:red, 255; green, 255; blue, 255 }  ,fill opacity=1 ] (410.98,228.99) -- (434.63,205.33) -- (449.25,205.33) -- (449.25,220.78) -- (425.6,244.43) -- (410.98,244.43) -- cycle ; \draw   (449.25,205.33) -- (425.6,228.99) -- (410.98,228.99) ; \draw   (425.6,228.99) -- (425.6,244.43) ;
%Shape: Cube [id:dp8941065814151745] 
\draw  [fill={rgb, 255:red, 255; green, 255; blue, 255 }  ,fill opacity=1 ] (429.42,228.99) -- (453.08,205.33) -- (467.7,205.33) -- (467.7,220.78) -- (444.04,244.43) -- (429.42,244.43) -- cycle ; \draw   (467.7,205.33) -- (444.04,228.99) -- (429.42,228.99) ; \draw   (444.04,228.99) -- (444.04,244.43) ;
%Shape: Cube [id:dp12133972287131023] 
\draw  [fill={rgb, 255:red, 255; green, 255; blue, 255 }  ,fill opacity=1 ] (447.87,228.99) -- (471.53,205.33) -- (486.15,205.33) -- (486.15,220.78) -- (462.49,244.43) -- (447.87,244.43) -- cycle ; \draw   (486.15,205.33) -- (462.49,228.99) -- (447.87,228.99) ; \draw   (462.49,228.99) -- (462.49,244.43) ;
%Shape: Cube [id:dp21632192124633232] 
\draw  [fill={rgb, 255:red, 255; green, 255; blue, 255 }  ,fill opacity=1 ] (465.84,228.99) -- (489.5,205.33) -- (504.12,205.33) -- (504.12,220.78) -- (480.46,244.43) -- (465.84,244.43) -- cycle ; \draw   (504.12,205.33) -- (480.46,228.99) -- (465.84,228.99) ; \draw   (480.46,228.99) -- (480.46,244.43) ;
%Shape: Cube [id:dp4671718328759742] 
\draw  [fill={rgb, 255:red, 255; green, 255; blue, 255 }  ,fill opacity=1 ] (410.98,210.74) -- (434.63,187.09) -- (449.25,187.09) -- (449.25,202.53) -- (425.6,226.19) -- (410.98,226.19) -- cycle ; \draw   (449.25,187.09) -- (425.6,210.74) -- (410.98,210.74) ; \draw   (425.6,210.74) -- (425.6,226.19) ;
%Shape: Cube [id:dp03657141776935058] 
\draw  [fill={rgb, 255:red, 255; green, 255; blue, 255 }  ,fill opacity=1 ] (429.42,210.74) -- (453.08,187.09) -- (467.7,187.09) -- (467.7,202.53) -- (444.04,226.19) -- (429.42,226.19) -- cycle ; \draw   (467.7,187.09) -- (444.04,210.74) -- (429.42,210.74) ; \draw   (444.04,210.74) -- (444.04,226.19) ;
%Shape: Cube [id:dp5792620082185109] 
\draw  [fill={rgb, 255:red, 255; green, 255; blue, 255 }  ,fill opacity=1 ] (447.87,210.74) -- (471.53,187.09) -- (486.15,187.09) -- (486.15,202.53) -- (462.49,226.19) -- (447.87,226.19) -- cycle ; \draw   (486.15,187.09) -- (462.49,210.74) -- (447.87,210.74) ; \draw   (462.49,210.74) -- (462.49,226.19) ;
%Shape: Cube [id:dp9897324295187186] 
\draw  [fill={rgb, 255:red, 255; green, 255; blue, 255 }  ,fill opacity=1 ] (465.84,210.74) -- (489.5,187.09) -- (504.12,187.09) -- (504.12,202.53) -- (480.46,226.19) -- (465.84,226.19) -- cycle ; \draw   (504.12,187.09) -- (480.46,210.74) -- (465.84,210.74) ; \draw   (480.46,210.74) -- (480.46,226.19) ;
%Shape: Cube [id:dp7982883030948849] 
\draw  [fill={rgb, 255:red, 255; green, 255; blue, 255 }  ,fill opacity=1 ] (338.25,192.5) -- (361.91,168.84) -- (376.53,168.84) -- (376.53,184.28) -- (352.87,207.94) -- (338.25,207.94) -- cycle ; \draw   (376.53,168.84) -- (352.87,192.5) -- (338.25,192.5) ; \draw   (352.87,192.5) -- (352.87,207.94) ;
%Shape: Cube [id:dp6401268927671004] 
\draw  [fill={rgb, 255:red, 92; green, 92; blue, 92 }  ,fill opacity=1 ] (356.7,192.5) -- (380.36,168.84) -- (394.98,168.84) -- (394.98,184.28) -- (371.32,207.94) -- (356.7,207.94) -- cycle ; \draw   (394.98,168.84) -- (371.32,192.5) -- (356.7,192.5) ; \draw   (371.32,192.5) -- (371.32,207.94) ;
%Shape: Cube [id:dp6310280328780073] 
\draw  [fill={rgb, 255:red, 92; green, 92; blue, 92 }  ,fill opacity=1 ] (375.15,192.5) -- (398.8,168.84) -- (413.42,168.84) -- (413.42,184.28) -- (389.77,207.94) -- (375.15,207.94) -- cycle ; \draw   (413.42,168.84) -- (389.77,192.5) -- (375.15,192.5) ; \draw   (389.77,192.5) -- (389.77,207.94) ;
%Shape: Cube [id:dp47611051598181064] 
\draw  [fill={rgb, 255:red, 92; green, 92; blue, 92 }  ,fill opacity=1 ] (393.12,192.5) -- (416.77,168.84) -- (431.39,168.84) -- (431.39,184.28) -- (407.73,207.94) -- (393.12,207.94) -- cycle ; \draw   (431.39,168.84) -- (407.73,192.5) -- (393.12,192.5) ; \draw   (407.73,192.5) -- (407.73,207.94) ;
%Shape: Cube [id:dp9226811176010099] 
\draw  [fill={rgb, 255:red, 255; green, 255; blue, 255 }  ,fill opacity=1 ] (338.25,174.25) -- (361.91,150.59) -- (376.53,150.59) -- (376.53,166.04) -- (352.87,189.69) -- (338.25,189.69) -- cycle ; \draw   (376.53,150.59) -- (352.87,174.25) -- (338.25,174.25) ; \draw   (352.87,174.25) -- (352.87,189.69) ;
%Shape: Cube [id:dp19892628827468428] 
\draw  [fill={rgb, 255:red, 92; green, 92; blue, 92 }  ,fill opacity=1 ] (356.7,174.25) -- (380.36,150.59) -- (394.98,150.59) -- (394.98,166.04) -- (371.32,189.69) -- (356.7,189.69) -- cycle ; \draw   (394.98,150.59) -- (371.32,174.25) -- (356.7,174.25) ; \draw   (371.32,174.25) -- (371.32,189.69) ;
%Shape: Cube [id:dp4573418701988108] 
\draw  [fill={rgb, 255:red, 92; green, 92; blue, 92 }  ,fill opacity=1 ] (375.15,174.25) -- (398.8,150.59) -- (413.42,150.59) -- (413.42,166.04) -- (389.77,189.69) -- (375.15,189.69) -- cycle ; \draw   (413.42,150.59) -- (389.77,174.25) -- (375.15,174.25) ; \draw   (389.77,174.25) -- (389.77,189.69) ;
%Shape: Cube [id:dp09767191064323733] 
\draw  [fill={rgb, 255:red, 92; green, 92; blue, 92 }  ,fill opacity=1 ] (393.12,174.25) -- (416.77,150.59) -- (431.39,150.59) -- (431.39,166.04) -- (407.73,189.69) -- (393.12,189.69) -- cycle ; \draw   (431.39,150.59) -- (407.73,174.25) -- (393.12,174.25) ; \draw   (407.73,174.25) -- (407.73,189.69) ;
%Shape: Cube [id:dp05174903625816629] 
\draw  [fill={rgb, 255:red, 255; green, 255; blue, 255 }  ,fill opacity=1 ] (338.25,156) -- (361.91,132.35) -- (376.53,132.35) -- (376.53,147.79) -- (352.87,171.45) -- (338.25,171.45) -- cycle ; \draw   (376.53,132.35) -- (352.87,156) -- (338.25,156) ; \draw   (352.87,156) -- (352.87,171.45) ;
%Shape: Cube [id:dp6328464402983587] 
\draw  [fill={rgb, 255:red, 92; green, 92; blue, 92 }  ,fill opacity=1 ] (356.7,156) -- (380.36,132.35) -- (394.98,132.35) -- (394.98,147.79) -- (371.32,171.45) -- (356.7,171.45) -- cycle ; \draw   (394.98,132.35) -- (371.32,156) -- (356.7,156) ; \draw   (371.32,156) -- (371.32,171.45) ;
%Shape: Cube [id:dp14841634536355053] 
\draw  [fill={rgb, 255:red, 92; green, 92; blue, 92 }  ,fill opacity=1 ] (375.15,156) -- (398.8,132.35) -- (413.42,132.35) -- (413.42,147.79) -- (389.77,171.45) -- (375.15,171.45) -- cycle ; \draw   (413.42,132.35) -- (389.77,156) -- (375.15,156) ; \draw   (389.77,156) -- (389.77,171.45) ;
%Shape: Cube [id:dp30721769172161895] 
\draw  [fill={rgb, 255:red, 92; green, 92; blue, 92 }  ,fill opacity=1 ] (393.12,156) -- (416.77,132.35) -- (431.39,132.35) -- (431.39,147.79) -- (407.73,171.45) -- (393.12,171.45) -- cycle ; \draw   (431.39,132.35) -- (407.73,156) -- (393.12,156) ; \draw   (407.73,156) -- (407.73,171.45) ;
%Shape: Cube [id:dp2578528119954442] 
\draw  [fill={rgb, 255:red, 255; green, 255; blue, 255 }  ,fill opacity=1 ] (338.25,137.76) -- (361.91,114.1) -- (376.53,114.1) -- (376.53,129.54) -- (352.87,153.2) -- (338.25,153.2) -- cycle ; \draw   (376.53,114.1) -- (352.87,137.76) -- (338.25,137.76) ; \draw   (352.87,137.76) -- (352.87,153.2) ;
%Shape: Cube [id:dp3317668849083417] 
\draw  [fill={rgb, 255:red, 255; green, 255; blue, 255 }  ,fill opacity=1 ] (356.7,137.76) -- (380.36,114.1) -- (394.98,114.1) -- (394.98,129.54) -- (371.32,153.2) -- (356.7,153.2) -- cycle ; \draw   (394.98,114.1) -- (371.32,137.76) -- (356.7,137.76) ; \draw   (371.32,137.76) -- (371.32,153.2) ;
%Shape: Cube [id:dp0711967967617364] 
\draw  [fill={rgb, 255:red, 255; green, 255; blue, 255 }  ,fill opacity=1 ] (375.15,137.76) -- (398.8,114.1) -- (413.42,114.1) -- (413.42,129.54) -- (389.77,153.2) -- (375.15,153.2) -- cycle ; \draw   (413.42,114.1) -- (389.77,137.76) -- (375.15,137.76) ; \draw   (389.77,137.76) -- (389.77,153.2) ;
%Shape: Cube [id:dp4442992468446201] 
\draw  [fill={rgb, 255:red, 255; green, 255; blue, 255 }  ,fill opacity=1 ] (393.12,137.76) -- (416.77,114.1) -- (431.39,114.1) -- (431.39,129.54) -- (407.73,153.2) -- (393.12,153.2) -- cycle ; \draw   (431.39,114.1) -- (407.73,137.76) -- (393.12,137.76) ; \draw   (407.73,137.76) -- (407.73,153.2) ;
%Shape: Cube [id:dp02422075239708077] 
\draw  [fill={rgb, 255:red, 255; green, 255; blue, 255 }  ,fill opacity=1 ] (410.98,192.5) -- (434.63,168.84) -- (449.25,168.84) -- (449.25,184.28) -- (425.6,207.94) -- (410.98,207.94) -- cycle ; \draw   (449.25,168.84) -- (425.6,192.5) -- (410.98,192.5) ; \draw   (425.6,192.5) -- (425.6,207.94) ;
%Shape: Cube [id:dp5246534281201924] 
\draw  [fill={rgb, 255:red, 255; green, 255; blue, 255 }  ,fill opacity=1 ] (429.42,192.5) -- (453.08,168.84) -- (467.7,168.84) -- (467.7,184.28) -- (444.04,207.94) -- (429.42,207.94) -- cycle ; \draw   (467.7,168.84) -- (444.04,192.5) -- (429.42,192.5) ; \draw   (444.04,192.5) -- (444.04,207.94) ;
%Shape: Cube [id:dp971684631303857] 
\draw  [fill={rgb, 255:red, 255; green, 255; blue, 255 }  ,fill opacity=1 ] (447.87,192.5) -- (471.53,168.84) -- (486.15,168.84) -- (486.15,184.28) -- (462.49,207.94) -- (447.87,207.94) -- cycle ; \draw   (486.15,168.84) -- (462.49,192.5) -- (447.87,192.5) ; \draw   (462.49,192.5) -- (462.49,207.94) ;
%Shape: Cube [id:dp6330511404674] 
\draw  [fill={rgb, 255:red, 255; green, 255; blue, 255 }  ,fill opacity=1 ] (465.84,192.5) -- (489.5,168.84) -- (504.12,168.84) -- (504.12,184.28) -- (480.46,207.94) -- (465.84,207.94) -- cycle ; \draw   (504.12,168.84) -- (480.46,192.5) -- (465.84,192.5) ; \draw   (480.46,192.5) -- (480.46,207.94) ;
%Shape: Cube [id:dp5408040065048967] 
\draw  [fill={rgb, 255:red, 255; green, 255; blue, 255 }  ,fill opacity=1 ] (410.98,174.25) -- (434.63,150.59) -- (449.25,150.59) -- (449.25,166.04) -- (425.6,189.69) -- (410.98,189.69) -- cycle ; \draw   (449.25,150.59) -- (425.6,174.25) -- (410.98,174.25) ; \draw   (425.6,174.25) -- (425.6,189.69) ;
%Shape: Cube [id:dp5616578644646302] 
\draw  [fill={rgb, 255:red, 255; green, 255; blue, 255 }  ,fill opacity=1 ] (429.42,174.25) -- (453.08,150.59) -- (467.7,150.59) -- (467.7,166.04) -- (444.04,189.69) -- (429.42,189.69) -- cycle ; \draw   (467.7,150.59) -- (444.04,174.25) -- (429.42,174.25) ; \draw   (444.04,174.25) -- (444.04,189.69) ;
%Shape: Cube [id:dp6141980404626759] 
\draw  [fill={rgb, 255:red, 255; green, 255; blue, 255 }  ,fill opacity=1 ] (447.87,174.25) -- (471.53,150.59) -- (486.15,150.59) -- (486.15,166.04) -- (462.49,189.69) -- (447.87,189.69) -- cycle ; \draw   (486.15,150.59) -- (462.49,174.25) -- (447.87,174.25) ; \draw   (462.49,174.25) -- (462.49,189.69) ;
%Shape: Cube [id:dp5510440617325265] 
\draw  [fill={rgb, 255:red, 255; green, 255; blue, 255 }  ,fill opacity=1 ] (465.84,174.25) -- (489.5,150.59) -- (504.12,150.59) -- (504.12,166.04) -- (480.46,189.69) -- (465.84,189.69) -- cycle ; \draw   (504.12,150.59) -- (480.46,174.25) -- (465.84,174.25) ; \draw   (480.46,174.25) -- (480.46,189.69) ;
%Shape: Cube [id:dp3862056075367579] 
\draw  [fill={rgb, 255:red, 255; green, 255; blue, 255 }  ,fill opacity=1 ] (410.98,156) -- (434.63,132.35) -- (449.25,132.35) -- (449.25,147.79) -- (425.6,171.45) -- (410.98,171.45) -- cycle ; \draw   (449.25,132.35) -- (425.6,156) -- (410.98,156) ; \draw   (425.6,156) -- (425.6,171.45) ;
%Shape: Cube [id:dp8142510949361561] 
\draw  [fill={rgb, 255:red, 255; green, 255; blue, 255 }  ,fill opacity=1 ] (429.42,156) -- (453.08,132.35) -- (467.7,132.35) -- (467.7,147.79) -- (444.04,171.45) -- (429.42,171.45) -- cycle ; \draw   (467.7,132.35) -- (444.04,156) -- (429.42,156) ; \draw   (444.04,156) -- (444.04,171.45) ;
%Shape: Cube [id:dp2149435904180943] 
\draw  [fill={rgb, 255:red, 255; green, 255; blue, 255 }  ,fill opacity=1 ] (447.87,156) -- (471.53,132.35) -- (486.15,132.35) -- (486.15,147.79) -- (462.49,171.45) -- (447.87,171.45) -- cycle ; \draw   (486.15,132.35) -- (462.49,156) -- (447.87,156) ; \draw   (462.49,156) -- (462.49,171.45) ;
%Shape: Cube [id:dp11194658086740494] 
\draw  [fill={rgb, 255:red, 255; green, 255; blue, 255 }  ,fill opacity=1 ] (465.84,156) -- (489.5,132.35) -- (504.12,132.35) -- (504.12,147.79) -- (480.46,171.45) -- (465.84,171.45) -- cycle ; \draw   (504.12,132.35) -- (480.46,156) -- (465.84,156) ; \draw   (480.46,156) -- (480.46,171.45) ;
%Shape: Cube [id:dp7193098755281542] 
\draw  [fill={rgb, 255:red, 255; green, 255; blue, 255 }  ,fill opacity=1 ] (410.98,137.76) -- (434.63,114.1) -- (449.25,114.1) -- (449.25,129.54) -- (425.6,153.2) -- (410.98,153.2) -- cycle ; \draw   (449.25,114.1) -- (425.6,137.76) -- (410.98,137.76) ; \draw   (425.6,137.76) -- (425.6,153.2) ;
%Shape: Cube [id:dp40465602423487623] 
\draw  [fill={rgb, 255:red, 255; green, 255; blue, 255 }  ,fill opacity=1 ] (429.42,137.76) -- (453.08,114.1) -- (467.7,114.1) -- (467.7,129.54) -- (444.04,153.2) -- (429.42,153.2) -- cycle ; \draw   (467.7,114.1) -- (444.04,137.76) -- (429.42,137.76) ; \draw   (444.04,137.76) -- (444.04,153.2) ;
%Shape: Cube [id:dp25101061162681293] 
\draw  [fill={rgb, 255:red, 255; green, 255; blue, 255 }  ,fill opacity=1 ] (447.87,137.76) -- (471.53,114.1) -- (486.15,114.1) -- (486.15,129.54) -- (462.49,153.2) -- (447.87,153.2) -- cycle ; \draw   (486.15,114.1) -- (462.49,137.76) -- (447.87,137.76) ; \draw   (462.49,137.76) -- (462.49,153.2) ;
%Shape: Cube [id:dp12988349997069237] 
\draw  [fill={rgb, 255:red, 255; green, 255; blue, 255 }  ,fill opacity=1 ] (465.84,137.76) -- (489.5,114.1) -- (504.12,114.1) -- (504.12,129.54) -- (480.46,153.2) -- (465.84,153.2) -- cycle ; \draw   (504.12,114.1) -- (480.46,137.76) -- (465.84,137.76) ; \draw   (480.46,137.76) -- (480.46,153.2) ;

%Shape: Rectangle [id:dp8361624255988072] 
\draw   (263.88,303.5) -- (277.25,303.5) -- (277.25,317.62) -- (263.88,317.62) -- cycle ;
%Shape: Rectangle [id:dp1157346875026748] 
\draw   (208.75,303.5) -- (222.12,303.5) -- (222.12,317.62) -- (208.75,317.62) -- cycle ;
%Shape: Rectangle [id:dp32795877398719564] 
\draw   (227.13,303.5) -- (240.49,303.5) -- (240.49,317.62) -- (227.13,317.62) -- cycle ;
%Shape: Rectangle [id:dp15680997373652228] 
\draw   (245.51,303.5) -- (258.87,303.5) -- (258.87,317.62) -- (245.51,317.62) -- cycle ;
%Shape: Rectangle [id:dp29653980593795715] 
\draw   (172,303.5) -- (185.36,303.5) -- (185.36,317.62) -- (172,317.62) -- cycle ;
%Shape: Rectangle [id:dp8700756716785643] 
\draw   (190.38,303.5) -- (203.74,303.5) -- (203.74,317.62) -- (190.38,317.62) -- cycle ;
%Shape: Rectangle [id:dp846059596981555] 
\draw   (282.26,303.5) -- (295.62,303.5) -- (295.62,317.62) -- (282.26,317.62) -- cycle ;
%Shape: Rectangle [id:dp4692167631375115] 
\draw   (300.64,303.5) -- (314,303.5) -- (314,317.62) -- (300.64,317.62) -- cycle ;
%Shape: Rectangle [id:dp8913736663390259] 
\draw   (263.88,322.04) -- (277.25,322.04) -- (277.25,336.16) -- (263.88,336.16) -- cycle ;
%Shape: Rectangle [id:dp5096163891029191] 
\draw  [fill={rgb, 255:red, 74; green, 74; blue, 74 }  ,fill opacity=1 ] (208.75,322.04) -- (222.12,322.04) -- (222.12,336.16) -- (208.75,336.16) -- cycle ;
%Shape: Rectangle [id:dp5516593360781625] 
\draw  [fill={rgb, 255:red, 74; green, 74; blue, 74 }  ,fill opacity=1 ] (227.13,322.04) -- (240.49,322.04) -- (240.49,336.16) -- (227.13,336.16) -- cycle ;
%Shape: Rectangle [id:dp36587134883673955] 
\draw   (245.51,322.04) -- (258.87,322.04) -- (258.87,336.16) -- (245.51,336.16) -- cycle ;
%Shape: Rectangle [id:dp46807070414643004] 
\draw   (172,322.04) -- (185.36,322.04) -- (185.36,336.16) -- (172,336.16) -- cycle ;
%Shape: Rectangle [id:dp8333607908242668] 
\draw  [fill={rgb, 255:red, 0; green, 0; blue, 0 }  ,fill opacity=0.67 ] (190.38,322.04) -- (203.74,322.04) -- (203.74,336.16) -- (190.38,336.16) -- cycle ;
%Shape: Rectangle [id:dp01819666017492616] 
\draw   (282.26,322.04) -- (295.62,322.04) -- (295.62,336.16) -- (282.26,336.16) -- cycle ;
%Shape: Rectangle [id:dp1462203211002704] 
\draw   (300.64,322.04) -- (314,322.04) -- (314,336.16) -- (300.64,336.16) -- cycle ;
%Shape: Rectangle [id:dp4115733246343811] 
\draw   (263.88,340.57) -- (277.25,340.57) -- (277.25,354.7) -- (263.88,354.7) -- cycle ;
%Shape: Rectangle [id:dp788910570445768] 
\draw  [fill={rgb, 255:red, 74; green, 74; blue, 74 }  ,fill opacity=1 ] (208.75,340.57) -- (222.12,340.57) -- (222.12,354.7) -- (208.75,354.7) -- cycle ;
%Shape: Rectangle [id:dp03868575827398968] 
\draw  [fill={rgb, 255:red, 74; green, 74; blue, 74 }  ,fill opacity=1 ] (227.13,340.57) -- (240.49,340.57) -- (240.49,354.7) -- (227.13,354.7) -- cycle ;
%Shape: Rectangle [id:dp0474139210754585] 
\draw   (245.51,340.57) -- (258.87,340.57) -- (258.87,354.7) -- (245.51,354.7) -- cycle ;
%Shape: Rectangle [id:dp768730372684556] 
\draw   (172,340.57) -- (185.36,340.57) -- (185.36,354.7) -- (172,354.7) -- cycle ;
%Shape: Rectangle [id:dp04782840834887603] 
\draw  [fill={rgb, 255:red, 74; green, 74; blue, 74 }  ,fill opacity=1 ] (190.38,340.57) -- (203.74,340.57) -- (203.74,354.7) -- (190.38,354.7) -- cycle ;
%Shape: Rectangle [id:dp22110163440721609] 
\draw   (282.26,340.57) -- (295.62,340.57) -- (295.62,354.7) -- (282.26,354.7) -- cycle ;
%Shape: Rectangle [id:dp4974059220139866] 
\draw   (300.64,340.57) -- (314,340.57) -- (314,354.7) -- (300.64,354.7) -- cycle ;
%Shape: Rectangle [id:dp5625018886346362] 
\draw   (263.88,359.11) -- (277.25,359.11) -- (277.25,373.23) -- (263.88,373.23) -- cycle ;
%Shape: Rectangle [id:dp34772714521605397] 
\draw  [fill={rgb, 255:red, 74; green, 74; blue, 74 }  ,fill opacity=1 ] (208.75,359.11) -- (222.12,359.11) -- (222.12,373.23) -- (208.75,373.23) -- cycle ;
%Shape: Rectangle [id:dp16989062841672053] 
\draw  [fill={rgb, 255:red, 74; green, 74; blue, 74 }  ,fill opacity=1 ] (227.13,359.11) -- (240.49,359.11) -- (240.49,373.23) -- (227.13,373.23) -- cycle ;
%Shape: Rectangle [id:dp3857712823528894] 
\draw   (245.51,359.11) -- (258.87,359.11) -- (258.87,373.23) -- (245.51,373.23) -- cycle ;
%Shape: Rectangle [id:dp16404156398971192] 
\draw   (172,359.11) -- (185.36,359.11) -- (185.36,373.23) -- (172,373.23) -- cycle ;
%Shape: Rectangle [id:dp33200050879271203] 
\draw  [fill={rgb, 255:red, 74; green, 74; blue, 74 }  ,fill opacity=1 ] (190.38,359.11) -- (203.74,359.11) -- (203.74,373.23) -- (190.38,373.23) -- cycle ;
%Shape: Rectangle [id:dp683563807974535] 
\draw   (282.26,359.11) -- (295.62,359.11) -- (295.62,373.23) -- (282.26,373.23) -- cycle ;
%Shape: Rectangle [id:dp22701451187838861] 
\draw   (300.64,359.11) -- (314,359.11) -- (314,373.23) -- (300.64,373.23) -- cycle ;
%Shape: Rectangle [id:dp1656459264997432] 
\draw   (263.88,376.77) -- (277.25,376.77) -- (277.25,390.89) -- (263.88,390.89) -- cycle ;
%Shape: Rectangle [id:dp4754385248743638] 
\draw   (208.75,376.77) -- (222.12,376.77) -- (222.12,390.89) -- (208.75,390.89) -- cycle ;
%Shape: Rectangle [id:dp5340230294459751] 
\draw   (227.13,376.77) -- (240.49,376.77) -- (240.49,390.89) -- (227.13,390.89) -- cycle ;
%Shape: Rectangle [id:dp516572410818321] 
\draw   (245.51,376.77) -- (258.87,376.77) -- (258.87,390.89) -- (245.51,390.89) -- cycle ;
%Shape: Rectangle [id:dp5709403299889431] 
\draw   (172,376.77) -- (185.36,376.77) -- (185.36,390.89) -- (172,390.89) -- cycle ;
%Shape: Rectangle [id:dp6567830733071638] 
\draw   (190.38,376.77) -- (203.74,376.77) -- (203.74,390.89) -- (190.38,390.89) -- cycle ;
%Shape: Rectangle [id:dp38933615089695084] 
\draw   (282.26,376.77) -- (295.62,376.77) -- (295.62,390.89) -- (282.26,390.89) -- cycle ;
%Shape: Rectangle [id:dp05428311527883234] 
\draw   (300.64,376.77) -- (314,376.77) -- (314,390.89) -- (300.64,390.89) -- cycle ;
%Shape: Rectangle [id:dp7907034923069554] 
\draw   (263.88,395.3) -- (277.25,395.3) -- (277.25,409.43) -- (263.88,409.43) -- cycle ;
%Shape: Rectangle [id:dp6479585629095648] 
\draw   (208.75,395.3) -- (222.12,395.3) -- (222.12,409.43) -- (208.75,409.43) -- cycle ;
%Shape: Rectangle [id:dp8086115869410486] 
\draw   (227.13,395.3) -- (240.49,395.3) -- (240.49,409.43) -- (227.13,409.43) -- cycle ;
%Shape: Rectangle [id:dp057794021519072425] 
\draw   (245.51,395.3) -- (258.87,395.3) -- (258.87,409.43) -- (245.51,409.43) -- cycle ;
%Shape: Rectangle [id:dp4959704574379633] 
\draw   (172,395.3) -- (185.36,395.3) -- (185.36,409.43) -- (172,409.43) -- cycle ;
%Shape: Rectangle [id:dp896766136251604] 
\draw   (190.38,395.3) -- (203.74,395.3) -- (203.74,409.43) -- (190.38,409.43) -- cycle ;
%Shape: Rectangle [id:dp5832507676281988] 
\draw   (282.26,395.3) -- (295.62,395.3) -- (295.62,409.43) -- (282.26,409.43) -- cycle ;
%Shape: Rectangle [id:dp1918801375480681] 
\draw   (300.64,395.3) -- (314,395.3) -- (314,409.43) -- (300.64,409.43) -- cycle ;
%Shape: Rectangle [id:dp46339319038156] 
\draw   (263.88,413.84) -- (277.25,413.84) -- (277.25,427.96) -- (263.88,427.96) -- cycle ;
%Shape: Rectangle [id:dp2302640067888866] 
\draw   (208.75,413.84) -- (222.12,413.84) -- (222.12,427.96) -- (208.75,427.96) -- cycle ;
%Shape: Rectangle [id:dp13578181647021115] 
\draw   (227.13,413.84) -- (240.49,413.84) -- (240.49,427.96) -- (227.13,427.96) -- cycle ;
%Shape: Rectangle [id:dp6979263623276024] 
\draw   (245.51,413.84) -- (258.87,413.84) -- (258.87,427.96) -- (245.51,427.96) -- cycle ;
%Shape: Rectangle [id:dp8773246337400837] 
\draw   (172,413.84) -- (185.36,413.84) -- (185.36,427.96) -- (172,427.96) -- cycle ;
%Shape: Rectangle [id:dp1988045428150138] 
\draw   (190.38,413.84) -- (203.74,413.84) -- (203.74,427.96) -- (190.38,427.96) -- cycle ;
%Shape: Rectangle [id:dp20923374962283137] 
\draw   (282.26,413.84) -- (295.62,413.84) -- (295.62,427.96) -- (282.26,427.96) -- cycle ;
%Shape: Rectangle [id:dp2864065534185878] 
\draw   (300.64,413.84) -- (314,413.84) -- (314,427.96) -- (300.64,427.96) -- cycle ;
%Shape: Rectangle [id:dp721907468540768] 
\draw   (263.88,432.38) -- (277.25,432.38) -- (277.25,446.5) -- (263.88,446.5) -- cycle ;
%Shape: Rectangle [id:dp18499468209604886] 
\draw   (208.75,432.38) -- (222.12,432.38) -- (222.12,446.5) -- (208.75,446.5) -- cycle ;
%Shape: Rectangle [id:dp46932148173819255] 
\draw   (227.13,432.38) -- (240.49,432.38) -- (240.49,446.5) -- (227.13,446.5) -- cycle ;
%Shape: Rectangle [id:dp8115573640619611] 
\draw   (245.51,432.38) -- (258.87,432.38) -- (258.87,446.5) -- (245.51,446.5) -- cycle ;
%Shape: Rectangle [id:dp07540376701712415] 
\draw   (172,432.38) -- (185.36,432.38) -- (185.36,446.5) -- (172,446.5) -- cycle ;
%Shape: Rectangle [id:dp49759788994093035] 
\draw   (190.38,432.38) -- (203.74,432.38) -- (203.74,446.5) -- (190.38,446.5) -- cycle ;
%Shape: Rectangle [id:dp04893453686763749] 
\draw   (282.26,432.38) -- (295.62,432.38) -- (295.62,446.5) -- (282.26,446.5) -- cycle ;
%Shape: Rectangle [id:dp8090312314125241] 
\draw   (300.64,432.38) -- (314,432.38) -- (314,446.5) -- (300.64,446.5) -- cycle ;

%Straight Lines [id:da3950159133054303] 
\draw    (339,294.29) -- (478.12,294.1) ;
\draw [shift={(481.12,294.1)}, rotate = 179.92] [fill={rgb, 255:red, 0; green, 0; blue, 0 }  ][line width=0.08]  [draw opacity=0] (10.72,-5.15) -- (0,0) -- (10.72,5.15) -- (7.12,0) -- cycle    ;
\draw [shift={(336,294.3)}, rotate = 359.92] [fill={rgb, 255:red, 0; green, 0; blue, 0 }  ][line width=0.08]  [draw opacity=0] (8.93,-4.29) -- (0,0) -- (8.93,4.29) -- cycle    ;
%Straight Lines [id:da6195382124737101] 
\draw    (321.02,138.3) -- (322.09,279.1) ;
\draw [shift={(322.12,282.1)}, rotate = 269.56] [fill={rgb, 255:red, 0; green, 0; blue, 0 }  ][line width=0.08]  [draw opacity=0] (10.72,-5.15) -- (0,0) -- (10.72,5.15) -- (7.12,0) -- cycle    ;
\draw [shift={(321,135.3)}, rotate = 89.56] [fill={rgb, 255:red, 0; green, 0; blue, 0 }  ][line width=0.08]  [draw opacity=0] (8.93,-4.29) -- (0,0) -- (8.93,4.29) -- cycle    ;
%Straight Lines [id:da7167588830783582] 
\draw    (174,459.29) -- (313.12,459.1) ;
\draw [shift={(316.12,459.1)}, rotate = 179.92] [fill={rgb, 255:red, 0; green, 0; blue, 0 }  ][line width=0.08]  [draw opacity=0] (10.72,-5.15) -- (0,0) -- (10.72,5.15) -- (7.12,0) -- cycle    ;
\draw [shift={(171,459.3)}, rotate = 359.92] [fill={rgb, 255:red, 0; green, 0; blue, 0 }  ][line width=0.08]  [draw opacity=0] (8.93,-4.29) -- (0,0) -- (8.93,4.29) -- cycle    ;
%Straight Lines [id:da13762486486844394] 
\draw    (155.02,310.3) -- (156.09,451.1) ;
\draw [shift={(156.12,454.1)}, rotate = 269.56] [fill={rgb, 255:red, 0; green, 0; blue, 0 }  ][line width=0.08]  [draw opacity=0] (10.72,-5.15) -- (0,0) -- (10.72,5.15) -- (7.12,0) -- cycle    ;
\draw [shift={(155,307.3)}, rotate = 89.56] [fill={rgb, 255:red, 0; green, 0; blue, 0 }  ][line width=0.08]  [draw opacity=0] (8.93,-4.29) -- (0,0) -- (8.93,4.29) -- cycle    ;
%Straight Lines [id:da026477221699595654] 
\draw    (101.07,313.82) -- (325,78) ;
\draw [shift={(99,316)}, rotate = 313.52] [fill={rgb, 255:red, 0; green, 0; blue, 0 }  ][line width=0.08]  [draw opacity=0] (10.72,-5.15) -- (0,0) -- (10.72,5.15) -- (7.12,0) -- cycle    ;
%Straight Lines [id:da43773941489089974] 
\draw    (313,79) -- (324,90) ;
%Straight Lines [id:da496555094531699] 
\draw    (135,269) -- (146,280) ;
%Straight Lines [id:da8041490031414376] 
\draw    (288,106) -- (299,117) ;


% Text Node
\draw (410.5,307.75) node  [font=\small] [align=left] {\begin{minipage}[lt]{37.4pt}\setlength\topsep{0pt}
{\small Spatial}
\end{minipage}};
% Text Node
\draw (307.5,207.75) node  [font=\small,rotate=-270.23] [align=left] {\begin{minipage}[lt]{37.4pt}\setlength\topsep{0pt}
{\small Spatial}
\end{minipage}};
% Text Node
\draw (248.5,474.75) node  [font=\small] [align=left] {\begin{minipage}[lt]{37.4pt}\setlength\topsep{0pt}
{\small Spatial}
\end{minipage}};
% Text Node
\draw (142.5,378.75) node  [font=\small,rotate=-270.23] [align=left] {\begin{minipage}[lt]{37.4pt}\setlength\topsep{0pt}
{\small Spatial}
\end{minipage}};
% Text Node
\draw (90,248.4) node [anchor=north west][inner sep=0.75pt]    {$t_{p+1}$};
% Text Node
\draw (259,85.4) node [anchor=north west][inner sep=0.75pt]    {$t_{p}$};
% Text Node
\draw (285,56.4) node [anchor=north west][inner sep=0.75pt]    {$t_{0}$};
% Text Node
\draw (211.27,180.01) node  [font=\small,rotate=-315.04] [align=left] {\begin{minipage}[lt]{69.63pt}\setlength\topsep{0pt}
Temporal axis\\
\end{minipage}};
% Text Node
\draw (345.72,346) node [anchor=north west][inner sep=0.75pt]   [align=left] {Newly \\acquired \\data};
% Text Node
\draw (340,430) node [anchor=north west][inner sep=0.75pt]    {$\{x^i_l \}_{i=1}^{n}$};
% Text Node
\draw (533.56,147) node [anchor=north west][inner sep=0.75pt]   [align=left] {Previous \\data};
% Text Node
\draw (540,200) node [anchor=north west][inner sep=0.75pt]    {$\{\mathbf{x}^i \}_{i=1}^{n}$};
\end{tikzpicture}}

}
\caption{\small \acs{SAR} data representation including both previous and recently obtained images. The local neighborhood of size $n$ is denoted by gray pixels (sliding window). }
\label{fig:datacube}
\end{figure}
\vspace{-5pt} 
\subsection{Background}
\label{sssec:background}
Taking into account the phase closure property of the InSAR stack, the covariance matrix of SAR images adheres to the following structure: 
\vspace{-7pt} 
\begin{equation}
\label{cov_mat_struc}
   \mathbf{\tilde{\Sigma}} = \mathbf{\tilde{\Psi}} \odot \mathbf{\tilde{w}}_{\mathbf{\theta}}\mathbf{\tilde{w}}_{\mathbf{\theta}}^H
\end{equation}
\vspace{-2pt} 
where the symbol $\odot$ represents the element-wise (Hadamard) multiplication, the exponent $^H$ is the transposed and complex conjugated operator (Hermitian), $\mathbf{\tilde{\Psi}}$ is the real core of the covariance matrix and $\mathbf{\tilde{w}}_{\mathbf{\theta}}$ denotes the vector of the exponential of the complex phases ($\mathbf{\tilde{w}}_{\mathbf{\theta}}=[e^{j \theta_0}, \dots,e^{j \theta_{l}}]$). The main idea of \acs{PL} is to estimate $\mathbf{\tilde{\Sigma}}$ which amounts to estimate $\mathbf{\tilde{\Psi}}$ and $\mathbf{\tilde{w}}_{\mathbf{\theta}}$ since the covariance matrix is connected to the unknown coherence matrix and phases, according to (\ref{cov_mat_struc}). 
\acs{PL} algorithms are summarized in \citep{10261889} and compared mathematically in \citep{cao2015mathematical}.  
The standard \acs{PL} consists of $2$ steps : \textit{i}) computing the \ac{SCM} over a local patch of the image and then \textit{ii}) considering the plug-in estimate of the coherence matrix $\mathbf{\tilde{\Psi}} = |\acs{SCM}|$ and solving the following optimization problem
\begin{gather}
\label{eq:classic_phaselinking}
    \begin{array}{cl}
    \underset{\mathbf{\tilde{w}}_{\mathbf{\theta}}}{\rm minimize}
    & 
    \mathcal{L}(|\acs{SCM}| \odot \mathbf{\tilde{w}}_{\mathbf{\theta}}\mathbf{\tilde{w}}_{\mathbf{\theta}}^H)
    \end{array}
\end{gather}
where $\mathcal{L}$ corresponds to the negative log-likelihood function of the data following the zero mean \ac{CCG} distribution.
This $2$-step approach relies on the plug-in estimate of the coherence matrix which renders it non optimal due to the bias associated with this plug-in. That is why \citep{vu2022new, vu2023robust} proposed to estimate jointly $\mathbf{\tilde{\Psi}}$ and $\mathbf{\tilde{w}}_{\mathbf{\theta}}$. 
Therefore, the optimization problem transforms into
\begin{equation}
\label{eq:MLE_phaselinking}
    \begin{array}{cl}
    \underset{\mathbf{\tilde{w}}_{\mathbf{\theta}}}{\rm minimize}
    & 
    \mathcal{L}(\mathbf{\tilde{\Psi}} \odot \mathbf{\tilde{w}}_{\mathbf{\theta}}\mathbf{\tilde{w}}_{\mathbf{\theta}}^H)
    \\
    \text{ subject to}
    & 
    \mathbf{\tilde{\Psi}} \; \text{real, } \; \mathbf{\tilde{w}}_{\theta} \in \mathbb{T}_{l} \text{,} \; , \; \theta_1 = 0 %\mathbf{\tilde{w}}_0 = 1 
    \end{array}
\end{equation} 
where $\mathbb{T}_l = \left\{\mathbf{\tilde{w}} \in \mathbb{C}^{l} | \, |[\tilde{w}]_i| = 1, \forall i \in [1, l]\right\}$ is the $l$-torus of phase only complex vectors.
\vspace{-5pt} 
\subsection{Covariance matrix structure with new data}
\label{sssec:cov_structure}
The hermitian structured covariance matrix, given in (\ref{cov_mat_struc}), can be rewritten as 
\begin{equation}
\label{cov_mat_struc_block}
    \mathbf{\tilde{\Sigma}} = \left( \begin{array}{c}
        \begin{tabular}{ccc}
        \multicolumn{2}{c}{\multirow{2}{*}{\huge{\Sigma}}} &  \\
            \multicolumn{2}{c}{} & $w_{\theta_{l}}^* diag(\mathbf{w}_{\theta})\boldsymbol{\gamma}^T $  \\
                &     $\boldsymbol{\gamma} diag(\mathbf{w}_{\theta})^H w_{\theta_{l}}$    & $\gamma_{l}$\\ % \exp^{(j \theta_{l})}
        \end{tabular}
        \end{array} \right)
\end{equation}
where the exponent $ ^*$ is the conjugated operator, $\mathbf{\Sigma}$ denotes the previously estimated covariance matrix between the previous \ac{SAR} images, $\boldsymbol{\gamma}$ corresponds to the coherence vector between the newly acquired data and the previous ones, $\gamma_{l}$ represents the variance of the newly acquired data,  and $w_{\theta_{l}}$ is the exponential of the phase of the latest data. We note that $3$ parameters associated with the new image are unknown and the remaining are estimated based on the methodology presented in \citep{vu2023robust}. These parameters will be represented by hats to indicate that they have already been estimated.
\vspace{-5pt} 
\subsection{\ac{MLE} problem}
\label{sssec:MLE_prob}
Considering the covariance matrix structure in (\ref{cov_mat_struc_block}) and assuming that $\{\mathbf{\tilde{x}}^i\}_{i=1}^n$ follows a \acs{CCG} distribution, the associated negative log-likelihood for the entire data set, can be expressed as:
\vspace{-20pt} 
\begin{equation}
\label{log_likelihood_eq}
\begin{aligned}
    \mathcal{L}_G(\boldsymbol{\gamma}, \gamma_{l}, w_{\theta_{l}}) &= - \sum_{i=1}^n \mathcal{L}_G^i(x_{p+1}^i | \mathbf{x}^i; \boldsymbol{\gamma}, \gamma_{l}, w_{\theta_{l}}) + \mathcal{L}_G^i(\mathbf{x}^i) \\
\end{aligned}
\end{equation}
According to \citep{anderson1958introduction}, % \citep{petersen2008matrix},
$x_{p+1}^i|\mathbf{x}^i \sim \mathcal{CN}(\mu_x^i, \sigma^2_x)$ 
where \\ 
$\mu_x^i = w_{\theta_{l}} \boldsymbol{\gamma} \text{diag}(\hat{\mathbf{w}}_{\theta})^H\hat{\mathbf{\Sigma}}^{-1} \mathbf{x}^i$ \\
and $\sigma^2_x = \gamma_l - \boldsymbol{\gamma} \text{diag}(\hat{\mathbf{w}}_{\theta})^H \hat{\mathbf{\Sigma}}^{-1}  \text{diag}(\hat{\mathbf{w}}_{\theta}^H) \boldsymbol{\gamma}^T$, and the negative log-likelihood in (\ref{log_likelihood_eq}), can be formulated as 

\begin{equation}
\label{log_likelihood_eq_final}
\begin{aligned}
    \mathcal{L}_G(\boldsymbol{\gamma}, \gamma_{l}, w_{\theta_{l}}) &\propto n \log\left( v \right) + \sum_{i=1}^n \frac{y^{i*} y^i}{v}.
\end{aligned}
\end{equation}

where $y^i = x_{l}^i - w_{\theta_{l}} \, \boldsymbol{\gamma} \, \text{diag}(\hat{\mathbf{w}}_{\theta})^H \hat{\mathbf{\Sigma}}^{-1} \mathbf{x}^i$ \\
and $v = \gamma_{l} - \boldsymbol{\gamma} \, \text{diag}(\hat{\mathbf{w}}_{\theta})^H \hat{\mathbf{\Sigma}}^{-1} \text{diag}(\hat{\mathbf{w}}_{\theta}) \boldsymbol{\gamma}^T$

In this work, we propose to estimate simultaneously the coherence and the new phase difference using the covariance matrix structure (\ref{cov_mat_struc_block}), 
\vspace{-10pt} 
\begin{equation}
\begin{aligned}
\min_{\boldsymbol{\gamma}, \gamma_{l}, \theta_{l}} \quad & \mathcal{L}_G(\boldsymbol{\gamma}, \gamma_{l}, w_{\theta_{l}}) \\
\text{subject to} \quad & \boldsymbol{\gamma}, \gamma_{l} \; \text{real}, \; |w_{\theta_{l}}| = 1 , \; \theta_1 = 0
%& \gamma_{l} \; \text{real} \\
\end{aligned}
\end{equation}
The optimization of $\mathcal{L}_G$, defined in (\ref{log_likelihood_eq}), will be addressed in a unified manner using a \acs{BCD} algorithm. The main idea of this algorithm involves estimating each parameter iteratively  while fixing the others. Thus, each update corresponds to an optimization sub-problem (cf. Algorithm \ref{Algorithm:algo_BCD}). 
\vspace{-10pt} 
\subsubsection*{Update $\boldsymbol{\gamma}$}
Let us start by updating $\boldsymbol{\gamma}$ by solving the following sub-problem %as in (\ref{Equation:problem2})
\begin{equation}
\label{Equation:problem2}
\min_{\boldsymbol{\gamma}} \quad  \mathcal{L}_G(\boldsymbol{\gamma}) \quad 
\text{s.t.} \quad  \boldsymbol{\gamma} \; \text{real}
\end{equation}
This optimization can be analytically solved as %results is an analytical solution for the update as
\begin{equation}
\label{Equation:gamma_vector}
\resizebox{\columnwidth}{!}{$
\small \boldsymbol{\gamma} = \small \left(\sum_{i=1}^n w_{\theta_{l}}^* x_{l}^i \mathbf{L}^i - w_{\theta_{l}} x_{l}^{i*} \mathbf{L}^{i*}  \right) . \left(\sum_{i=1}^n \mathbf{M}^i + \mathbf{M}^{i*} \right)^{-1}
$}
\end{equation}
where $\small \mathbf{L}^i = \small \mathbf{x}^{iH} \hat{\mathbf{\Sigma}}^{-1} \text{diag}(\hat{\mathbf{w}}_{\theta})$ and $\small \mathbf{M}^i = \small \mathbf{L}^{iH} \mathbf{L}^i$
%\small \text{diag}(\mathbf{w}_{\theta})^H \mathbf{C}^{-1}\mathbf{x^i}\mathbf{x^{iH}} \mathbf{C}^{-1} \text{diag}(\mathbf{w}_{\theta})$
\vspace{-10pt} 
\subsubsection*{Update $\gamma_{l}$}
$\gamma_{l}$ is updated by minimizing $\mathcal{L}_G$ while fixing $\boldsymbol{\gamma}$ and $w_{\theta_l}$
\vspace{-10pt} 
\begin{equation}
\min_{\gamma_{l}} \quad  \mathcal{L}_G(\gamma_{l}) \quad 
\text{s.t.} \quad  \gamma_{l} \; \text{real}
\end{equation} 
\vspace{-5pt} 
The variance of the newly acquired data is calculated as
\begin{equation}
\label{Equation:gamma_scalar}
\resizebox{\columnwidth}{!}{$
\small \gamma_l = \small \frac{1}{n} \sum_{i=1}^{n} (x_{l}^i - w_{\theta_{l}} \boldsymbol{\gamma} \mathbf{L}^{iH})^*(x_{l}^i - w_{\theta_{l}} \boldsymbol{\gamma} \mathbf{L}^{iH}) \\
    +  \boldsymbol{\gamma} \mathbf{N} \boldsymbol{\gamma}^T 
$}
\end{equation}
\vspace{-10pt} 
where $\small \mathbf{N} = \small \text{diag}(\hat{\mathbf{w}}_{\theta})^H \hat{\mathbf{\Sigma}}^{-1} \text{diag}(\hat{\mathbf{w}}_{\theta})$
\subsubsection*{Update $w_{\theta_{l}}$}
The phase difference of the newly acquired \acs{SAR} image is obtained by solving the following sub-problem
\begin{equation}
\label{Equation:problem4}
\begin{aligned}
\min_{w_{\theta_l}} \quad & \mathcal{L}_G(w_{\theta_{l}}) \quad \text{s.t.} \quad & |w_{\theta_{l}}| = 1, \; \theta_1 = 0 
\end{aligned}
\end{equation}
As a result, the phase difference of the newly acquired images takes the following form
\vspace{-10pt} 
\begin{equation}
\label{Equation:new_phase}
\small w_{\theta_l} = \small \frac{\Big( \big(\sum_{i=1}^n x_{l}^{i} \mathbf{L}^i \boldsymbol{\gamma}^T \big) . \big(\sum_{i=1}^n \boldsymbol{\gamma} \mathbf{M}^i \boldsymbol{\gamma}^T \big)^{-1}\Big)}{|| \Big( \big(\sum_{i=1}^n x_{l}^{i} \mathbf{L}^i \boldsymbol{\gamma}^T \big) . \big(\sum_{i=1}^n \boldsymbol{\gamma} \mathbf{M}^i \boldsymbol{\gamma}^T \big)^{-1}\Big) ||}
\end{equation}
\begin{algorithm}
\setstretch{0.5} 
\caption{\ac{BCD} algorithm }
\label{Algorithm:algo_BCD}
\begin{algorithmic}[1]
\State \textbf{Input}: Samples $\{\mathbf{\tilde{x}}^i\}_{i=1}^n$, $\hat{\mathbf{\Sigma}}$, $\text{diag}(\hat{\mathbf{w}}_{\theta})$
% \State Computation of $\mathbf{S} =$ \acs{SCM}
\Repeat
    \State Update of $\boldsymbol{\gamma}$ with (\ref{Equation:gamma_vector})
    \State Update of $\gamma_{l}$ with (\ref{Equation:gamma_scalar})
    \State Update of $w_{\theta_{l}}$ with (\ref{Equation:new_phase})
\Until{convergence}
\State \textbf{Output}: $\boldsymbol{\gamma}$, $\gamma_{l}$ and $w_{\theta_{l}}$
\end{algorithmic}
\end{algorithm}
